\documentclass[a4paper]{article}
\begin{document}


\section{The best/worst/most surprising things about this course unit}
The best thing about this course for me is that getting to know my group members, and working, communicating with them. I find out that when solving a problem by myself, I might be struggling with something that is not important or less important for a long time which makes my working efficiency very low. Nevertheless, working with my group members can reduce a lot of inefficient work for me, and I can get feedback from my group members. When we have group meetings, we can discuss the different idea, present our thought to each other, analyze the advantage and disadvantage of our idea, and trying to merge the advantages of them to get the best concept out of them. Also, we separate our works so that everyone can contribute to the field that we are good at, and everyone can balance the work we get. When working as a team, I can learn more things than working along and be more efficient.

\section{The most important things that you have learnt during this course unit}
The most important things that I have learned during this course unit are always asking questions when I am confused about something, and develop my work base on the feedback I got. I was used to focusing on things I don't understand by myself, searching for material online, and read some books about it, and I find it extremely hard to get useful information and wasted a lot of time while I could just ask professors, teaching assistants and tutor for help. Most of the time, I was just struggling with something that doesn't work at all. Finally, I realize that asking questions is the most important thing for me to be more efficient, and it is the most efficient way to learn things.

\section{How I learn, manage my time, and organise my tasks}
For this course unit, I usually watch the course lectures first with the handouts when they are published, and do more research on the topic I have watched to get a better understanding. Next, I will finish the quiz, and using the lab instructions to complete the labs before Thursday every week. If I have some questions or problems with the lab, I will ask the teaching assistants during the online lab on Friday every week; else, I will show my works to the teaching assistants, get feedback from them, and develop my work base on the feedback I got. 
For the tutorial each week, checking the tutorial instruction on the blackboard is always the first thing I will do every Monday, to make sure I am clear on what I need to do for the next tutorial and having my work done before the tutorial on Tuesday. Next, during the tutorial, I will present my work with my team member to our tutorial, and get some feedback from him. Finally, I and my team member will have a zoom meeting to discuss the work we need to do and develop our work depending on the feedback we got.

\section{How did the pandemic/partial lockdown/working from home affect my teamwork? How did it affect my learning?}
The pandemic lockdown has a great impact on my teamwork and make things hard to achieve. Due to the lockdown, I and my team members can only communicate and work online. More over, the time difference between us makes things even harder because we are in different time zone. However, we tried a lot of ways of solving the problem. We use Whatsapp to discuss things like when we are meeting each other, what we need to do, and how we are going to do it; We find a lot of websites and applications that allow us to collaborate at the same time; We have group meetings every week to discuss, separate works so that we can balance everyone’s work, ask questions and give feedback to each other. After the works and communications we have done, I find out that I and our team did a great job.

It also affects my learning. I have to stay up late and change my habits to join live lectures and labs. But I am getting used to the way I learn. I download the notes and resources, go through them after the lecture every week. Then, I will spend one to two days to finish my lab exercises and quizzes. Finally, I will ask some questions I have on the online laboratory sessions and show my works to teaching assistants.

\section{How I have developed/changed since the start of the year?}
At the start of the year, I was panic about almost everything because of the pandemic lockdown and studying remotely. I don't know how to use the blackboard, how to manage my time because I am in a different time zone; I don't know what I need to do or where to find the resources I need, I even forget to watch the recorded lectures. At first, I don't know there are recorded lectures and handouts, and I just join the live lectures, therefore I find it very hard to understand what we are learning about and panic for a long time. After this semester, I find out that I can manage my time well, I can use blackboard fluently to find whatever I need and knowing a lot of websites and books to find more resources. All in all, I think I have become more and more used to my university life and I am confident that I will do better for the next semester.

\section{What changes do I need to make next semester and how do I plan to achieve these changes?}
What I need is to be more efficient in learning things and be a good time manager. To achieve these changes, I need to keep watching every week's recorded lectures, download and go through all the handouts, and then join the live lectures. After that, I will work on my lab exercises, and finding more resources to learn more. When I have problems or questions, I will go and research online first, if I still can not understand them, I will bring my questions with my work to the online laboratory sessions. Also, I will make a time table and a to-do list every week to keep reminding myself of what I need to do and when I need to do something to manage my time. I am sure I will do a great job next semester.


\end{document}